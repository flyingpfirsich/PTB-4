\chapter{Grundlagen}
    \section{Geschäftsprozessmanagement}
   - BPM includes concepts, meth-
ods, and techniques to support the design, administration, configuration, enact-
ment, and analysis of business processes
- Definition
- Aufgaben
- Bereiche
- Nutzung in der Industrie
- Stärken
- Schwächen
- Auf BPMN2.0 eingehen
- BPMNS als Softwarelösung für BPM
\subsection{BPM Lifecycle}
    \section{Robotic Process Automation}
- Definition
- RPA is an umbrella term for tools that operate on
the user interface of other computer systems in the way a human would do” [1].
RPA is an upcoming
- Use Cases
- Harmon [20] indicated that 30% of the surveyed practitioners 
would like to add some kind of RPA capabilities to their process modeling suite. 
-Probleme von RPA:
Despite all benefits, RPA has strong limitations: In order to identify and imple-
ment an RPA process, extensive process knowledge is required. Existing work
has shown that, if no such knowledge is available (e.g. no other systems for
gathering it are in place), the benefits of RPA are far less significant, as much
time and effort has to be put into gaining that knowledge [5, 6]
- - fehlende Infos über automation enactment(was soll automatisiert werden?)
- Ausnahmebehandlungen
- Automatisierung in die Organisation einbetten

    \section{BPMNS-RPA}
- Vorschlag, beide Technologien zu verbinden
- beide Technolgien haben Gemeinsamkeiten, beide bauen auf Prozesse auf, haben die gleichen Ziele
- sind im Moment jedoch völlig getrennt: Though these technologies are very often used separately, the authors from business practice [14, 36] strongly suggest combining both to gain even more business value. In a case of the lack of resources and/or time to completely implement BPMS, RPA can
4 be a valuable and relatively inexpensive tool to solve or complement some of the un-fulfilled goals.
- BPM kann Rahmen schaffen, damit RPA schneller skalieren kann
- BPMN Notation könnte Brücke bilden
- somit ist: As RPA systems can only automate processes on a low level
of abstraction, RPA processes can be considered activities of a parent business
process.
- kann Probleme von RPA lösen
BPMNS-RPA kann eine Mögllichkeit sein.
\subsection{BPMN-RPA Lifecycle}
\chapter{Marktübersicht}
- Reihe von Anbietern auf RPA und BPMN Seite,
- es wird untersucht, in wie fern die Anbieter Methoden beider Disziplinen vereinen.
- untersucht werden folgende:
\chapter{Fallstudie: BPMN Datenaustausch zwischen SAP Build Process Automation und SAP Signavio}

    - Überleitung zum Hauptteil der Studienarbeit
\section{SAP Signavio}
\section{SAP Build Process Automation}

\section{Problem Statement}
    - hier auf die angeführten Probleme aus Kapitel 1 eingehen
    - Personas vorstellen
    - User Demand angeben


   \section{Integration}
   - die stärkere kollaboration zwischen Signavio Process Manager und SBPA kann als Schritt in Richtung BPMS-RPA verstanden werden
   - damit besteht die Mögllichkeit für SAP, auf dem Gebiet vorreiter zu werden
   - Sie kann die in Kapitel 1 beschriebenen Probleme lösen
   - einen einheitlichen ende-zu-ende Prozess darstellen und rpa skalierbar machen
   - ein erster Versuch ist folgendes Feature:
   - es wird überprüft, in wie fern ein automatischer BPMN Datenaustausch zwischen Signavio und SBPA zu realisieren ist.
   - als MVP wird der manuelle BPMN import gesetzt
   - Nach Evaluation des MVPs sind weitere tiefergreifende Integrationen vorstellbar

   - hier auf den POC eingehen
    .Es wird eine Integration evaluiert, um signavio und sbpa stärker zu integrieren
\subsection{Technische Voraussetzung}
- sbpa hat eine workflow engine, die bpmn2.0 compliant ist, basiert auf xxx engine
- jedoch werden in der design time der Anwendung nicht alle shapes unterstützt.
-hier tabelle mit shapes einfügen

   - welche optionen werden evaluiert?
    - eine iflow Integration
    - einen bpmn Import
    - eine native Integration
    - auf den lifecycle eingehen
    - ea story erzählen
    
    \section{To-Be Modellierung}
    - User Journey?
    - man identifiziert einen manuellen Prozess in Signavio
    - Der Prozess zeigt ein hohes Automatisierungspotential an
    - man kann den Prozess zunächst manuell herunterladen
    - dann bei SBPA importieren
    - in SBPA anpassen, RPA-Bots, Connectoren, usw. einbinden
    - Prozess testen
    - man hat einen manuellen Prozess automatisiert, ohne ihn doppelt zu modellieren
    - UI Mockups einbinden
    \section[short]{SAP Enterprise Automation}
    - EA als RPA-BPMNS bestreben von SAP
\chapter{Ausblick}
    \section{Bewertung}
    - manueller Import immer noch zu aufwendig
    - nicht alle Artefakte lassen sich übertragen
        - man muss trotzdem viel in SBPA anpassen
        - darum ist keine Synchronisation möglich
    \section{Weiterentwicklung}
    \section{Weitere Integrationsszenarien}