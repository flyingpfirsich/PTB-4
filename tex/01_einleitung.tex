\chapter{Einleitung}
%- BPMN als reiferes Gebiet hilft Unternehmen beim erfassen, managen, analysieren von Prozessautomatisierung
%- Der Druck nach Automatisierung steigt, um wettbewerbsfähig zu bleiben
%- RPA versprach diese Automatisierung, der gewünscht Effekt blieb jedoch Ausblick
%- Das versprechen konnte nicht eingelöst werden, RPA hat einige Probleme
%- Da BPMN und RPA überschneidungspunkte haben, ist eine tiefere Integration der Fachgebiete vorstellbar
%- sie könnten voneinader profitieren, RPA könnte sein Versprechen von unternehmensweiter Automatisierung einlösen
%- Die Frage stellt sich, wie eine Integration aussehen könnte, was RPA benötigt, um den gewünschten Effekt zu erzielen
%- Kann eine Integration zwischen RPA und BPMN diesen Bedarf decken


Die Verwaltung und Optimierung von Geschäftsprozessen spielt eine essenzielle Rolle für Unternehmen. Hierfür hat sich \gls{bpm} als ausgereiftes Gebiet zur Modellierung, Erfassung und Analyse von Geschäftsprozessen etabliert. Zusätzlich steigt der Druck auf Unternehmen, Geschäftsprozesse weiter zu automatisieren.

\gls{rpa} wurde als vielversprechende Technologie eingeführt, um diese geforderte Automatisierung zu erfüllen. Viele Unternehmen mussten jedoch feststellen, dass \gls{rpa} zwar für bestimmte Automatisierungen geeignet ist, aber nicht in der Lage ist, komplexe Geschäftsprozesse ganzheitlich zu automatisieren \citep[S. 7]{Costa2022}. Zu den Problemen von \gls{rpa} gehören die mangelnde Skalierbarkeit und Flexibilität, sowie das fehlende Wissen, \gls{rpa} in die bestehende IT-Landschaft zu integrieren \citep{König2020RPA-BPMS}. Da \gls{bpm} und \gls{rpa} Gemeinsamkeiten aufweisen - beide agieren auf der Geschäftsprozessebene - ist eine tiefere Integration beider Technologien naheliegend. So könnten die zur Zeit unabhängig betrachteteten Technologien voneinander profitieren: \gls{bpm} könnte \gls{rpa} die notwendige Reife und Skalierbarkeit zur Verfügung stellen, während \gls{bpm} um Automatisierungsmöglichkeiten erweitert wird. 

In dieser Studienarbeit wird die Frage diskutiert, wie eine mögliche Integration zwischen \gls{rpa} und \gls{bpm} aussehen könnte, welche Bedingungen erfüllt sein müssen und welche Chancen sich daraus ergeben. Anschließend wird eine Marktübersicht gegeben, die beschreibt, wie heutige Anbieter diese Funktionalitäten in ihren Produkten miteinader integrieren. Danach wird eine konkrete Funktionalität vorgestellt und evaluiert, die eine Integration zwischen SAP Signavio (\gls{bpm}) und SAP Build Process Automation (\gls{rpa}) darstellt. 

